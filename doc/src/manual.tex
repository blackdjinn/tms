\documentclass[10pt,letterpaper,twoside]{book}
\author{Ryan M. Davis iconoclasmandheresy@gmail.com}
\title{TMS manual}
\usepackage{pxfonts}
\begin{document}
\pagestyle{empty}
\maketitle
\frontmatter
\pagestyle{headings}
\tableofcontents
\listoftables
\mainmatter
\part{User Guide}
\chapter{Input translation}
\begin{table}
\begin{tabular}{c p{'10cm'}}
char & Description\\
\hline
: & Maps to ``pose rest-of-line''\\
" & Maps to ``say rest-of-line''\\
; & Maps to ``pose -nospace rest-of-line''\\
\textbackslash & Maps to ``spoof rest-of-line''\\
@ & Evaluate in connection context\\
\# & Evaluate in location context\\
+ & Evaluate in character context (default)\\
\hline
\end{tabular}
\caption{Input syntax symbols}
\label{tab:inputsyntaxchars}
\end{table}
\part{Builder Guide}
\part{Coding}
\part{Admin Guide}
\part{Server Internals}
\chapter{Concepts}
Wherever possible, the built-in capabilities of the programming environment have been used.

Tcl's event loop provides sufficient concurrency, so it is used in place of threads.

The TinyMUD derived family of servers which this is designed to replace are typically augmented with a connection to a SQL server.
Instead, TMS uses SQL as the primary persistant data store.
The Tcl data access libraries are considered among the most robust available in resisting SQL-Injection attacks,
a critical concern for an environment designed to allow untrusted access.
\chapter{Structure}
Internally, the server is constructed using n-trees of objects.
Each node keeps a list of its children, and a pointer back up to its parent object.
Calls can propagate up and be delegated out as needed.
All internal nodes should be subclasses of 'handler'.
Ultimately, most leaves should be instances of 'connection' representing individual network connections.

An example of this is the 'broadcast' method.
The inherited operation of this method is to recursively invoke 'brodcast' on the parent until the root is reached,
The 'echo' method is then invoked which by inheritance invokes 'echo' on all the children.
\chapter{Shells}
A shell represents a context for parsing commands.
A shell must impliment the 'parse' method as something more than simply handing that off to its parent.
Each incoming line from a connection invokes the parent's 'parse' method, following the tree toward the root.
The message propagates through the tree until it reaches a shell.
At that point it is parsed in whatever way is context appropriate, and messages are dispatched back down the tree.
\section{Login Shell}
New incoming connections are assigned to an instance of loginshell.
It can perform some simple queries, or log into a game or chatroom.
\section{Chat Shell}
The chatshell impliments a basic instant-message like chatroom.
Everyone is in the same room if they are connected to this instance.
It was created for testing but since it is small and of potential use, it remains in the system.
\section{Game Shell}
Where a chatshell does little more than broadcast text to various connections, the game shell does a lot more.
Each incoming line is handed to a sandboxed safe tcl interpreter.
It impliments a simple REPL (Read Evaluate Print Loop) in the context of that interpreter.
From this point, the expressive power of the Tcl language can be accessed.
A handle to a private SQL data-store is also available, with a simple hierarchical store implimented on top of it.
%\backmatter
%\part{Appendix}
%\part{Index}
%\index
\end{document}
